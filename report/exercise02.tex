\section{Exercise 2}

\subsection{Part 1}
The \texttt{Vertex} struct consists of only a \texttt{position} field, which is a \texttt{vec4} compared to \texttt{vec2} from exercise 1. The \texttt{vec4} represents a position in a homogeneous coordinates, hence we are working in 3D space.

Since we are working in 3D, the \texttt{display} function now also computes projection matrices using either the \texttt{Ortho} or \texttt{Projection} function.

Another new function is the \texttt{glUniformMatrix4fv} which changes the the \texttt{projection} variable in the vertex shader (\texttt{const-shader.vert}).

\subsection{Part 2}
\fig{images/exer_2_2}

\subsubsection{a}
The \texttt{modelView} transformations were done in reverse order of the specification, due to \textit{column}-matrices.

\begin{lstlisting}
mat4 modelView(1.0f);
modelView *= Translate(0.0f, 3.0f, 0.0f);
modelView *= RotateY(30.0f);
modelView *= Scale(2.0f);
\end{lstlisting}


\subsubsection{b}
I used Translate, Scale and RotateY.

\subsubsection{c}
If Scale is done before Translate then one translation unit becomes twice the size, thus translating the object further.

\subsection{Part 3}

\subsubsection{a}
\fig[one-point perspective]{images/exer_2_3_front}

\begin{lstlisting}
vec4 eye(20.0f, 5.0f, 5.0f, 0);
vec4 up(0.0f, 1.0f, 0.0f, 0);
vec4 at(0.0f, 5.0f, 5.0f, 0);

mat4 modelView;

modelView *= LookAt(eye, at, up);
modelView *= Scale(4.0f);
\end{lstlisting}

\subsubsection{b}
\fig[two-point perspective]{images/exer_2_3_x}


\subsection{Part 4}

The \texttt{eye} is positioned on the diagonal of the cube and looks straight into it, thus creating the isometric view.

\begin{lstlisting}
vec4 eye(1.0f, 1.0f, 1.0f, 0);
vec4 up(0.0f, 0.0f, 1.0f, 0);
vec4 at(0.0f, 0.0f, 0.0f, 0);
\end{lstlisting}

\subsection{Part 5}

\subsubsection{a}
I rotated first then translate to eye position. The angle between \texttt{eye} and \texttt{at} is computed as:

\[
angle = acos\left( \frac{eye \bullet at}{\lvert eye \rvert \cdot \lvert at \rvert}  \right)
\]

\fig{images/exer_2_5_key2}{}

\begin{lstlisting}
vec4 eye = vec4(-.5, 1, 6, 1);
vec4 at = vec4(-2, 1, 0, 1);

float angle = acos((dot(eye, at)) / (length(eye) * length(at)));
float angleDeg = angle * 180.0f * M_1_PI;

view = mat4();
view *= RotateY(angleDeg - 90);
view *= Translate(.5, -1, -6);
\end{lstlisting}

\subsubsection{b}
\texttt{eye} is opposite of the translation in \texttt{key3}.
\begin{lstlisting}
vec4 up(0, 1, 0, 0);
vec4 eye(4, 1, 1, 0);
vec4 at = RotateY(-92) * eye;

view = LookAt(eye, at, up);
\end{lstlisting}

\subsubsection{c}

\begin{lstlisting}
view = mat4(1, 0, 0, 0,
			0, 1, 0, 0,
			0, 0, 1, 0,
			1, -1, -9, 1);
\end{lstlisting}


\subsection{Part 6}

\subsubsection{a}
The resulting matrix is the following
\[
\left[ {\begin{array}{cccc}
      1 & 0 & 0 & 0 \\
      0 & 0.894427299 & -0.44721365 & 0 \\
      0 & 0.44721359 & 0.89442718 & 0 \\
      0 & 0 & -6.70820332 & 1
    \end{array} } \right]
\]


\subsubsection{b}
The resulting matrix is the following
\[
\left[ {\begin{array}{cccc}
      1.73205078 & 0 & -1 & 0 \\
      0 & 2 & 0 & 0 \\
      1 & 0 & 1.73205078 & 0 \\
      0 & 3 & 0 & 1
    \end{array} } \right]
\]


%%% Local Variables:
%%% mode: latex
%%% TeX-master: "main"
%%% End:
