\section{Exercise 1}
\subsection{Part 1}
\fig{images/exer_1_1}
A brief explanation of each function:

\begin{description}
\item [display] Does the actual OpenGL drawing, by clearing the screen, setting the shader and drawing the triangles. It also swaps buffers, which is redundant since we're only interested

\item [reshape] Sets the viewport when the window is resized.

\item[loadBufferData] Loads the \texttt{vertexData} into the GPU

\item[loadShader] Loads the vertex and fragment shaders using an
  \texttt{Angel} function

\item[main] Initializes OpenGL, registers handlers for the display and reshape events, and loads the data via the above mentioned functions.

\end{description}
The \texttt{vertexData} for the rendered image is as follows

\begin{lstlisting}
const int NUMBER_OF_VERTICES = 6;
Vertex vertexData[NUMBER_OF_VERTICES] = {
	{ vec2( -0.5, -0.5 ), vec3( 1.0, 0.0, 0.0 ) },
	{ vec2(  0.5, -0.5 ), vec3( 0.0, 1.0, 0.0 ) },
	{ vec2(  0.5,  0.5 ), vec3( 0.0, 0.0, 1.0 ) },

	{ vec2( -0.5, -0.5 ), vec3( 1.0, 1.0, 1.0 ) },
	{ vec2(  0.5,  0.5 ), vec3( 1.0, 1.0, 1.0 ) },
	{ vec2( -0.5,  0.5 ), vec3( 1.0, 1.0, 1.0 ) }
}
\end{lstlisting}

\subsection{Part 2}
\fig{images/exer_1_2}

\subsubsection{a}

The functions' responsibilities are the same as in the previous exercise, but there is a new function \texttt{loadGeometry} which utilizes \texttt{loadBufferData} to create a rectangle and triangle vertex array buffer.

\subsubsection{b}
The below code includes the coloring of part C (red, green and blue vertices)
\begin{lstlisting}
// in loadGeometry:
vec3 redColor(1.0f, 0.0f, 0.0f);
vec3 greenColor(0.0f, 1.0f, 0.0f);
vec3 blueColor(0.0f, 0.0f, 1.0f);
Vertex triangleData[triangleSize] = {
	{ vec2(2.0, 2.0), redColor },
	{ vec2(5.0, 2.0), greenColor },
	{ vec2(3.5, 5.0), blueColor }
};

// Globally defined GLuint triangleVertexArrayBuffer
triangleVertexArrayBuffer = loadBufferData(triangleData, triangleSize);
\end{lstlisting}

\subsubsection{c}
We are using column matrices, so the x and y coordinates are located on the bottom of the transformation matrix.

\begin{lstlisting}
// in display:
mat4 modelViewTri = mat4(
    1, 0, 0, 0,
	0, 1, 0, 0,
	0, 0, 1, 0,
	6, 7, 0, 1
);

glUniformMatrix4fv(modelViewUniform, 1, GL_TRUE, modelViewTri);
glBindVertexArray(triangleVertexArrayBuffer);
glDrawArrays(GL_TRIANGLE_FAN, 0, triangleSize);
\end{lstlisting}

\subsubsection{d}
The rotation transformation matrix is constructed using \texttt{sin} and \texttt{cos}:

\begin{lstlisting}
// global
const GLfloat DEG_TO_RAD = M_PI / 180.0;

// in display:
GLfloat rot = -45.0 * DEG_TO_RAD;
mat4 modelViewRect = mat4(
    cos(rot), -sin(rot), 0, 0,
    sin(rot),  cos(rot), 0, 0,
    0, 0, 1, 0,
    0, 0, 0, 1
);

glUniformMatrix4fv(modelViewUniform, 1, GL_TRUE, modelViewRect);
glBindVertexArray(rectangleVertexArrayBuffer);
glDrawArrays(GL_TRIANGLE_FAN, 0, rectangleSize);
\end{lstlisting}

\subsection{Part 3}
\fig{images/exer_1_3}

\subsubsection{a}
\texttt{glDrawArrays} instructs OpenGL to draw the buffered vertex data linearly in the order it was loaded. \texttt{glDrawElements} takes an \texttt{indices} argument that holds a list of ordered indices of the vertices to draw. This means that vertices can be reused, which can make a huge difference for meshes that share a lot of vertices, but it also requires OpenGL to transfer more data to the GPU, which is expensive.

\subsubsection{b}
Consider the data \texttt{vertices = [A, B, C, D, E]}, and assume we are drawing with \texttt{glDrawArrays}:

\begin{description}
\item[\texttt{GL\_TRIANGLES}]
  will draw a triangle for every third vertex, discarding any remainders. \texttt{vertices} would be drawn as a single triangle \texttt{\{A,B,C\}} and \texttt{D,E} would be discarded.

\item[\texttt{GL\_TRIANGLE\_FAN}] will use the first vertex as anchor, and will draw a triangle for every other consecutive pair of vertices. \texttt{vertices} would thus be drawn as \texttt{\{A,B,C\}, \{A,C,D\}, \{A,D,E\}}.

\item[\texttt{GL\_TRIANGLE\_STRIP}]
   will draw a triangle for \textit{each} three consecutive vertices. \texttt{vertices} will be drawn as \texttt{\{A,B,C\}, \{B,C,D\}, \{C,D,E\}}.
\end{description}

\subsubsection{c}
This shows how to use \texttt{glDrawElements} to draw different images from the same vertex data, and how vertices are reused.

\begin{lstlisting}
// in display:
mat4 modelView;

// upper left
modelView = Translate(-7,+7,0);
glUniformMatrix4fv(modelViewUniform, 1, GL_TRUE, modelView);
glDrawArrays(GL_TRIANGLE_FAN, 0, rectangleSize);

// upper right
modelView = Translate(+7,+7,0);
glUniformMatrix4fv(modelViewUniform, 1, GL_TRUE, modelView);
GLuint indices[6] = {0,1,2,3,4,0};
glDrawElements(GL_TRIANGLES, 6, GL_UNSIGNED_INT, indices);

// bottom left
modelView = Translate(-7,-7,0);
glUniformMatrix4fv(modelViewUniform, 1, GL_TRUE, modelView);
GLuint indices_bl[6] = {2,3,4,0,1,5};
glDrawElements(GL_TRIANGLES, 6, GL_UNSIGNED_INT, indices_bl);

// bottom right
modelView = Translate(+7,-7,0);
glUniformMatrix4fv(modelViewUniform, 1, GL_TRUE, modelView);
GLuint indices_br[9] = {1,2,3, 0,1,5, 3,4,5};
glDrawElements(GL_TRIANGLES, 9, GL_UNSIGNED_INT, indices_br);
\end{lstlisting}


\subsection{Part 4}
\fig{images/exer_1_4}

To make this a generic circle creation i have made following function

\begin{lstlisting}
std::vector<Vertex> createCircleGeometry(GLfloat radius, vec2 origin, GLuint segments) {
	std::vector<Vertex> vertices;

	GLfloat segAngle = (2 * M_PI) / segments;
	for (int i = 0; i < segments; i++) {
		vertices.push_back({
				vec2(cos(i * segAngle) * radius + origin.x,
					 sin(i * segAngle) * radius + origin.y )
		});
	}
	return vertices;
}

void loadGeometry() {
	cirleVertexData = createCircleGeometry(0.8, vec2(0.0f, 0.0f), 10);
}
\end{lstlisting}

The data is buffered:
\begin{lstlisting}
// in loadBufferData:
glBufferData(GL_ARRAY_BUFFER, cirleVertexData.size() * sizeof(Vertex), &cirleVertexData[0], GL_STATIC_DRAW);
\end{lstlisting}

And drawn:
\begin{lstlisting}
// in display:
glBindVertexArray(vertexArrayObject);
glDrawArrays(GL_TRIANGLE_FAN, 0, cirleVertexData.size());
\end{lstlisting}

\subsection{Part 5}
\fig{images/exer_1_5}

I have made the image using two loops, one for the inner triangles and one for the outer. The inner loop simply rotates the \texttt{modelView} and draws the vertices with that transformation. The outer one is a little more complex as it must both translate, scale and rotate each triangle.

\begin{lstlisting}
int i;

// inner
modelView = Scale(1.0, -1.0, 1.0);
for (i = 0; i < 4; i++)
{
	modelView *= RotateZ(90.0f); // rotate 90 deg every iteration
	glUniformMatrix4fv(modelViewUniform, 1, GL_TRUE, modelView);
	glDrawArrays(GL_TRIANGLES, 0, NUMBER_OF_VERTICES);
}

// outer
for (i = 0; i < 4; i++)
{
	float scaleY = i % 2 == 0 ? -SQRT2 : SQRT2; // is right/left
	float trans = i > 1 ? -1.0f : 1.0f; // is bottom/left
	modelView = Translate(trans, trans, 0.0f);
	modelView *= Scale(SQRT2, scaleY, SQRT2);
	modelView *= RotateZ(45.0f + (float) i * 90.0f);
	glUniformMatrix4fv(modelViewUniform, 1, GL_TRUE, modelView);
	glDrawArrays(GL_TRIANGLES, 0, NUMBER_OF_VERTICES);
}
\end{lstlisting}


\subsection{Part 7}
The comments in the code should explain what's happening. The image is rendered with 5 subdivisions.

\fig{images/exer_1_7}

\begin{lstlisting}
void square(const vec2& a, const vec2& b, const vec2& c, const vec2& d) {
	// draw the square as two triangles across the diagonal
    points.push_back(a);
	points.push_back(b);
	points.push_back(c);

	points.push_back(a);
	points.push_back(c);
	points.push_back(d);
}

void divide_square(const vec2& a, const vec2& b, const vec2& c, const vec2& d, int count) {
    if (count > 0) {
		// side length of the *sub-square* of the current square
		const float d = length((b - a) / 3.0f);

		// utility vectors in the x and y directions used to computed
		// sub-squares easily
		const vec2 dx(d, 0.0f);
		const vec2 dy(0.0f, d);

		for (int y = 0; y < 3; y++) {
			for (int x = 0; x < 3; x++) {
				// leave the center square white
				if (x == 1 && y == 1) continue;

				// compute the sub-square coordinates using the (x,y)
				// pair and the side lengths defined earlier
				divide_square(a + x*dx     + y*dy,
							  a + (x+1)*dx + y*dy,
							  a + (x+1)*dx + (y+1)*dy,
							  a + x*dx     + (y+1)*dy,
							  count - 1);
			}
		}
    } else {
		// leaf, draw a black square
        square( a, b, c, d );
    }
}

// In the init function:
void init( void ) {
    vec2 vertices[4] = {
        vec2( -1.0, -1.0 ),
		vec2(  1.0, -1.0 ),
		vec2(  1.0,  1.0 ),
        vec2( -1.0,  1.0 )
    };

    // Subdivide the original square
    divide_square( vertices[0], vertices[1], vertices[2], vertices[3], NumTimesToSubdivide);
...
\end{lstlisting}


%%% Local Variables:
%%% mode: latex
%%% TeX-master: "main"
%%% End:
